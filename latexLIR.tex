%\documentclass[a4 paper,12pt]{report}
\documentclass[a4 paper,12pt,ukrainian]{report}
\usepackage[12pt]{extsizes}
\usepackage[utf8]{inputenc}
%\usepackage[utf8]{inputenc}
\usepackage[ukrainian]{babel}
\usepackage{amsfonts,amsmath}
%\usepackage{amsthm}
\usepackage{makeidx,bezier,latexsym,epsfig,layout}
\textheight 22.5 cm \textwidth 16 cm \topmargin -1 cm
\evensidemargin 0 cm \oddsidemargin 0 cm

\newtheorem{theorem}{\textbf{Теорема}}[chapter]
%\newtheorem{proof}{\textbf{Дов.:}}
\newtheorem{lema}[theorem]{\textbf{Лема}}
\newtheorem{determination}[theorem]{\textbf{Означення}}

\begin{document}
%!!!!!!!!!!!!!!!!!!!!!!!! титульна !!!!!!!!!!!!!!!!!!!!!
\titlepage
\begin{center}
\large {Міністерство освіти та науки України}\\
\large {Львівський національний університет імені Івана Франка}\\
\large {Факультет прикладної математики та інформатики}\\
\end{center}

\normalsize \vspace*{2cm}
\hspace*{11cm}Допущено до захисту\\
\hspace*{11cm}Завідувач кафедри\\
\hspace*{11cm}\underline{\hspace{3cm}}\\
\hspace*{11cm}проф. Хапко Р.С.\\
\hspace*{11cm}"\underline{\hspace{0.5cm}}"\underline{\hspace{2cm}} 2015 р.
\normalsize \vspace*{2cm}
\begin{center}
\large{Роман Андрій}\\
\large{Зварич Андрій}\\
\large{\textbf{Чисельне розв'язування мішаної для рівнняння Лапласа методом ІР}}\\
\large{Звіт}\\
\end{center}
\normalsize \vspace*{3cm}
\hspace*{11cm}Наукові консультанти\\
\hspace*{11cm}проф. Р. Хапко\\
\hspace*{11cm}ст.викл. Я. Гарасим\\
\hspace*{11cm}асист. В. Вавричук\\
\\
\\
\\
\\
\\
\\
\large\centerline{Львів 2015}
%!!!!!!!!!!!!!!!!!!!!!!!!сторінка змісту!!!!!!!!!!!!!!!!!!!!!
\tableofcontents
%!!!!!!!!!!!!!!!текст курсової!!!!!!!!!!!!!!!!!!!!!!!!
\chapter*{\bf{Вступ}}
\addcontentsline{toc}{chapter}{Вступ}
\hspace*{\parindent}Дослiдження граничних задач для рiвнянь в частинних похiдних є однiєю з
найважливiших сфер застосування методу iнтегральних рiвнянь. Цей метод має низку
незаперечних переваг у порiвняннi з iншими методами, а саме: зменшення розмiрностi
задачi на одиницю, застосовнiсть для областей з границями довiльної форми i частково
необмежених областей тощо. Саме тому цей метод з успiхом використовується для
вирiшення складних iнженерних задач - плоских i просторових, стацiонарних та
нестацiонарних, в механiцi руйнувань гiрських порiд, теорiї теплопровiдностi, а також
в рiзних задачах геофiзики, електродинамiки тощо.\\
\hspace*{\parindent}Метою даної роботи є навчитися чисельно розв’язувати крайовi задачi для
рiвняння Лапласа у обмеженій області. Планується довести єдинiсть класичного
розв’язку поставленої задачi. Будемо застосовувати непрямий метод iнтегральних рiвнянь. Також потрiбно
з’ясувати коректнiсть отриманого iнтегрального рiвняння в просторах Банаха, здiйснити
параметризацiю iнтегрального рiвняння i видiлення особливостi.\\
\hspace*{\parindent}Наближений розв’язок iнтегрального рiвняння будемо знаходити методом квадратур
з використанням тригонометричних квадратур. Потрiбно довести збiжнiсть методу i
знайти апрiорну оцiнку похибки, а також наближений розв’язок диференцiальної задачi.
\chapter{Загальні відомості}
\section{Постановка задачі}
\hspace*{\parindent}Нехай $D_0\subset \mathbb{R}^2$ - круг радіуса $R>0$ з границею $\Gamma_0$, $D_1\subset D_0$ і $D_2\subset D_0$ - області з границями $\Gamma_i\in C^2$, $i=1,2$. Причому $D_1\cap D_2=\{\emptyset\}$.\\ 
\hspace*{\parindent}Позначимо $D=D_0\setminus (\bar{D_1}\cup\bar{D_2})$. Необхідно знайти функцію $u:D \rightarrow \mathbb{R}$, яка задовольняє рівняння Лапласа
\begin{equation}
\Delta u=0\text{ в }D
\end{equation}
і граничні умови 
\begin{equation}
		u & = & f_2 \quad \mbox{на} \quad \Gamma_2, \\
		\frac{\partial u}{\partial \nu} & = & f_1\quad \mbox{на} \quad \Gamma_2,\\
		\frac{\partial u}{\partial \nu} & = & f_0\quad \mbox{на} \quad \Gamma_0,
\end{equation}

Тут $f_0$, $f_1$, $f_2$ - задані функції.\\
\begin{figure}[h]
\center{\includegraphics[scale=0.8]{pict1}}
\caption{Геометрія області}
\label{fig:image}
\end{figure}\\

%####################################################################################################################################################################################################################################################################################################################################################################################################################

\section{Єдиність класичного розв'язку задачі}
\begin{determination}
Двічі неперервно-диференційовна дійснозначна функція $u\in C^2(D)$, визначена в $D\subset\mathbb{R}^2$, називається гармонічною, якщо вона задовольняє рівняння Лапласа (1.1).
\end{determination}
\begin{theorem}
Задача (1.1)-(1.2) має щонайбільше один розв'язок.
\end{theorem}
\textbf{Дов.}(Від супротивного). Нехай існують дві гармонічні функції $u_1$, $u_2$, які є різними розв'язками задачі (1.1)-(1.2). Очевидно, що $u=u_1-u_2$ - також гармонічна функція, яка рівна нулю на границі $D$ (оскільки $u_1=u_2$ на $\partial D$).\\
\hspace*{\parindent}Використавши першу теорему Гріна [1]:
\begin{equation*}
\int\limits_{D}\left\{u \ \Delta v+gradu\cdot gradv\right\}dx=\int\limits_{\partial D}u\frac{\partial v}{\partial\nu}ds,
\end{equation*}
де $\nu$ - зовнішня нормаль. При $u=v$ маємо:
\begin{equation*}
u \ \Delta u=0 \quad (\Delta u=0\text{ в }D)
\end{equation*}
\begin{equation*}
u\frac{\partial v}{\partial\nu}=0 \quad (u=0\text{ на }\partial D)
\end{equation*}
\begin{equation*}
\Rightarrow\int\limits_{D}gradu\cdot gradu \ dx=0
\end{equation*}
\begin{equation*}
\int\limits_{D}\Bigl(\frac{\partial u}{\partial x_1},\frac{\partial u}{\partial x_2}\Bigr)\cdot\Bigl(\frac{\partial u}{\partial x_1},\frac{\partial u}{\partial x_2}\Bigr)dx=\int\limits_{D}\left\{\Bigl(\frac{\partial u}{\partial x_1}\Bigr)^2+\Bigl(\frac{\partial u}{\partial x_2}\Bigr)^2\right\}dx=0
\end{equation*}
\begin{equation*}
\Rightarrow \ \frac{\partial u}{\partial x_1}=0, \quad \frac{\partial u}{\partial x_2}=0 \ \Rightarrow \ u=const.
\end{equation*}
Оскільки на $\partial D \ u=0$, то: $ \ u\equiv 0 \ \Rightarrow \ u_1=u_2$ - суперечність. $\quad\quad\quad\Box$ 
\newpage



%####################################################################################################################################################################################################################################################################################################################################################################################################################


\section{Потенціали та їх властивості}
\begin{determination}
Нехай задана функція $\varphi\in C(\partial D)$, $D\subset\mathbb{R}^m$ - деяка область. Функції
\begin{equation*}
u(x) = \int\limits_{\partial{D}} \, \varphi(y)N(x,y) ds(y) , \quad x \in \mathbb{R}^m\setminus D
\end{equation*}
та
\begin{equation*}
\upsilon(x) = \int\limits_{\partial{D}} \, \varphi(y)\frac{\partial N(x,y)}{\partial\nu(y)} ds(y) , \quad x \in \mathbb{R}^m\setminus D
\end{equation*}
називають відповідно потенціалом простого шару і потенціалом подвійного шару з неперервною густиною $\varphi$.
\end{determination}
\hspace*{\parindent}Для розв'язування граничних задач за допомогою потенціалів, необхідно дослідити їхню поведінку на границі $\partial{D}$, тому доцільно ввести наступні твердження.
\begin{theorem}
Нехай $\partial{D}\in C^2$, $\varphi\in C(\partial{D})$. Тоді потенціал простого шару $u$ з густиною $\varphi$ є неперервним в $\mathbb{R}^m$. На границі маємо:
\begin{equation*}
u(x) = \int\limits_{\partial{D}} \, \varphi(y)N(x,y) ds(y) , \quad x \in \partial D,
\end{equation*}
де інтеграл існує як невласний.
\end{theorem}
\begin{theorem}
Нехай $\partial D\in C^2$ і $\varphi\in C(\partial D)$. Тоді потенціал подвійного шару $\upsilon$ з густиною $\varphi$ може бути неперервно продовжений з $D$ до $\bar{D}$ і з $\mathbb{R}^m\setminus\bar{D}$ до $\mathbb{R}^m\setminus D$ з граничними значеннями
\begin{equation*}
\upsilon_\pm(x) = \pm\frac{1}{2}\varphi(x)+\int\limits_{\partial{D}} \, \varphi(y)\frac{\partial N(x,y)}{\partial\nu(y)} ds(y) , \quad x \in \partial D,
\end{equation*}
де $\upsilon_\pm(x)=\lim\limits_{h \to 0}\upsilon(x\pm h\nu(x))$ і інтеграли існують як невласні. 
\end{theorem}
\begin{theorem}
Нехай $\partial D\in C^2$ і $\varphi\in C(\partial D)$. Тоді для потенціалу простого шару $u$ з густиною $\varphi$ маємо:
\begin{equation*}
\frac{\partial u_\pm}{\partial\nu}(x)=\mp\frac{1}{2}\varphi(x)+\int\limits_{\partial D}\varphi(y)\frac{\partial N(x,y)}{\partial\nu(x)} ds(y) , \quad x \in \partial D,
\end{equation*}
де $\frac{\partial u_\pm}{\partial\nu}(x)=\lim\limits_{h \to 0}\nu(x)\cdot gradu(x\pm h\nu(x))$ і інтеграли існують як невласні.
\end{theorem}
%####################################################################################################################################################################################################################################################################################################################################################################################################################



\chapter{Чисельне розв'язування крайової задачі}
\section{Інтегральне подання розв'язку}
\hspace*{\parindent}Фундаментальним розв'язком задачі (1.1)-(1.2) є функція Гріна задачі Неймана для рівняння Лапласа в крузі $D_{0}\subset{\mathbb{R}^2}$ з радіусом $R$ для усіх $x\not=y$ в $\bar{D_{0}}$ 
\begin{equation*}
N(x,y) = \frac{1}{2\pi} \ln{\frac{1}{|x-y|}} + \tilde{N}(x,y),
\end{equation*}
де
\begin{equation*}
\tilde{N}(x,y) = \frac{1}{2\pi}\ln{\bigg(\bigg|y-\frac{xR^2}{|x|^2}\bigg|\frac{|x|}{R^3}\bigg)}.
\end{equation*}
\hspace*{\parindent}Подамо розв'язок задачі (1.1)-(1.2) у вигляді:
\begin{equation}
\ u(x)=u_{1}(x)+u_{2}(x),
\end{equation}
де $u_{1}(x)$ - розв'язок задачі:
\begin{equation}
 \begin{cases}
   \Delta{u_{1}} = 0 \quad \textup{в} \quad D_{0},
   \\
   \frac{\partial u_{1}}{\partial\nu} = f_{0}  \quad \textup{на} \quad \Gamma_{0},
 \end{cases}
\end{equation}
$u_{2}(x)$ - розв'язок задачі:
\begin{equation}
 \begin{cases}
   \Delta{u_{2}} = 0 \quad \textup{в} \quad D_{0} \backslash({D_{1}}\cup {D_{2}})%D_1\bigcup D_2,
   \\
   u_{2}=0 \quad \textup{на} \quad \Gamma_{0},
	\\
   \frac{\partial u_{2}}{\partial\nu} = f_{1} - u_{1} \quad \textup{на} \quad \Gamma_{1},
	\\
   u_{2} = f_{2} - u_{1} \quad \textup{на} \quad \Gamma_{2},

 \end{cases}
\end{equation}
\hspace*{\parindent}Розв'язком задачі (2.2) є інтеграл вигляду:
\begin{equation}
u_{1}(x) = \int\limits_{\Gamma_{0}} \, f_{0}(y)N(x,y) ds(y). 
\end{equation}
\hspace*{\parindent}Розв'язок задачі (2.3):
\begin{equation}
u_{2}(x) = \int\limits_{\Gamma_{1}} \, \varphi_{1}(y)N(x,y)ds(y)+\int\limits_{\Gamma_{2}} \, \varphi_{2}(y)\frac{\partial N(x,y)}{\partial\nu}ds(y). 
\end{equation}
\hspace*{\parindent}Тоді розв'язок задачі (1.1)-(1.2) можна подати у вигляді лінійної комбінації потенціалів:
\begin{equation}
u(x) = \int\limits_{\Gamma_{1}} \, \varphi_{1}(y)N(x,y)ds(y)+\int\limits_{\Gamma_{2}} \, \varphi_{2}(y)\frac{\partial N(x,y)}{\partial\nu}ds(y)+
\end{equation}
\begin{equation*}
+\int\limits_{\Gamma_{0}} \, f_{0}(y)N(x,y) ds(y), \quad x \in D,
\end{equation*}
де густини $\varphi_{1}$, $\varphi_{2}$ визначаються з системи інтегральних рівнянь другого роду:
\begin{equation}\label{13}
 \left\{
\begin{array}{c}
   \displaystyle
\frac{1}{2}\varphi_{1}(x) + \int\limits_{\Gamma_1} \, \varphi_1(y)N(x,y)ds(y)+\int\limits_{\Gamma_2} \, \varphi_2(y)\frac{\partial N(x,y)}{\partial\nu}ds(y)=f_1(x)-\frac{\partial u_{1}(x)}{\partial\nu},\\ \quad x\in \Gamma_1,\\

	\displaystyle
  -\frac{1}{2}\varphi_{2}(x) + \int\limits_{\Gamma_1} \, \varphi_1(y)N(x,y)ds(y)+\int\limits_{\Gamma_2} \, \varphi_2(y)\frac{\partial N(x,y)}{\partial\nu}ds(y)=f_2(x)-u_{1}(x),\\ \quad x\in \Gamma_2,
 \end{array}
\right.
\end{equation}
де
\begin{equation*}
\omega(x) = -\int\limits_{\Gamma_0} \, f_0(y)\frac{\partial G(x,y)}{\partial \nu(y)} ds(y) , \quad x \in D .
\end{equation*}


%########################################################################################################################################################################################################################################################################################################################################################################################################################################################################
%ПОМІНЯТИ!!!!!!!!!!!!!!!!!!!!!!!!!!!!!!!!!!!!!!!!!!!!!!!!!!!!!!!!!!!!!!!!!!!!!!!!!!!!!!!!!!!!!!
\newpage
\section{Коректність інтегральних рівнянь у просторах Банаха}
\begin{determination}
Дійсна або комплекснозначна функція $f$, визначена на $D\subset\mathbb{R}^{m}$, називається рівномірно неперервною за Гьольдером з показником $0<\alpha\le 1$, якщо існує стала $C>0$:
\begin{equation*}
|f(x)-f(y)|\le C|x-y|^{\alpha}, \quad x,y\in D.
\end{equation*}
\hspace*{\parindent}Позначимо через $C^{0,\alpha}(D)$- лінійний простір всіх функцій, які є обмеженими і рівномірно неперервними. Простір $C^{0,\alpha}(D)$ є простором Гьольдера.
\end{determination}
\begin{theorem}
Простір Гьольдера $C^{0,\alpha}(D)$ є Банаховим простором з нормою
\begin{equation*}
\Vert f\Vert_{\alpha}:=\sup\limits_{x\in D}|f(x)|+\sup\limits_{\substack{x,y\in D, \\ x\neq y}}\frac{|f(x)-f(y)|}{|x-y|^{\alpha}}.
\end{equation*}
\end{theorem}
\hspace*{\parindent}Розглянемо оператор
\begin{equation*}
(S\varphi)(x)=\int\limits_{\Gamma}{\varphi(y)N(x,y)}ds(y), \quad x\in\Gamma.
\end{equation*}
\begin{theorem}
Оператор
\begin{equation*}
S:C^{0,\alpha}(\Gamma)\to C^{1,\alpha}(\Gamma)
\end{equation*}
є обмеженим і самоспряженим відносно відповідних дуальних систем.
\end{theorem}
\begin{theorem}
Для будь-якої функції $f\in C(\Gamma)$ існує єдине інтегральне рівняння першого роду задач Діріхле з єдиним розв'язком $\varphi\in C^{0,\alpha}(\Gamma)$. 
\end{theorem}
Отже, для нашого випадку справедливим буде наступне твердження.
\begin{theorem}
Для будь-яких $f_{1}$, $f_{2}$ існують єдиним чином визначені густини $\varphi_{1}$, $\varphi_{2}$. Тобто існує єдиний розв'язок системи (2.7), який неперервно залежить від вхідних даних.
\end{theorem}


%################################################################################################################################################################################################################################################################################################################################################################################################################################################################
\section{Параметризація та виділення особливості}
\hspace*{\parindent}Нехай границі $\Gamma_{0}$, $\Gamma_{1}$ та $\Gamma_{2}\in C^{2}$ задаються параметрично:
\begin{equation*}
\Gamma_i :=\{x_i(t) = (x_i_1(t),x_i_2(t)), t \in [0,2\pi]\}, i=0,1,2.
\end{equation*}
Тут $x_{i}: [0,2\pi]\to\mathbb{R}^{2}\in C^{2}[0,2\pi]-2\pi$-періодична функція, для якої $|x_{i}'(t)|>0$, $t\in[0,2\pi]$.\\ 
\hspace*{\parindent}Тоді систему інтегральних рівнянь (2.7) можна подати в параметричному вигляді:
\begin{equation}
\left\{
\begin{array}{c}
\displaystyle
\frac{1}{2\pi}\mu_1(t) + \frac{1}{2\pi}\int\limits_{0}^{2\pi} \, \mu_1 (\tau)H_{1}(t,\tau)d\tau+\frac{1}{2\pi}\int\limits_{0}^{2\pi} \, \mu_2 (\tau)\frac{\partial L_1(t,\tau)}{\partial\nu}d\tau=g_1(t),\\ t\in [0, 2\pi]\\
\displaystyle
-\frac{1}{2\pi}\mu_2(t) + \frac{1}{2\pi}\int\limits_{0}^{2\pi} \, \mu_1 (\tau)L_{2}(t,\tau)d\tau+\frac{1}{2\pi}\int\limits_{0}^{2\pi} \, \mu_2 (\tau)\frac{\partial H_2(t,\tau)}{\partial\nu}d\tau=g_2(t),\\ t\in [0, 2\pi]
\end{array}
\right.
\end{equation}
де 
\begin{equation*}
\mu_{i}(t)=\varphi(x_{i}(t))|x_{i}'(t)|-2\pi\textup{-періодична функція}, \quad i=1,2,
\end{equation*}
\begin{equation*}
g_{1}(t)=f_{i}(x_{i}(t))-u_1(x_{i}(t)),\\
g_{2}(t)=f_{i}(x_{i}(t))-u_1(x_{i}(t)),\\
\end{equation*}
\begin{equation*}
L_{1}(t,\tau)=2\pi G(x_1(t),x_2(\tau))
\end{equation*}
\begin{equation*}
L_{2}(t,\tau)=2\pi G(x_2(t),x_1(\tau))
\end{equation*}
\begin{equation*}
H_{i}(t,\tau)=2\pi G(x_i(t),x_i(\tau)), \quad i=1,2.
\end{equation*}
\hspace*{\parindent}Ядра $H_{i}(t,\tau)$ мають логарифмічну особливість при $t=\tau$. Для виділення цієї особливості, подамо ядра в такому вигляді:
\begin{equation*}
H_{i}(t,\tau)=H_{i1}(t,\tau)\ln{\frac{4}{e}\sin^2\frac{t-\tau}{2}}+H_{i2}(t,\tau), \quad i=1,2,
\end{equation*}
де
\begin{equation*}
H_{i2}(t,\tau)=\left\{
\begin{array}{l}
\displaystyle
\frac{1}{2}\ln{\frac{4\sin^2\frac{t-\tau}{2}}{e|x_{i}(t)-x_{i}(\tau)|^2}}+\tilde{G}(x_{i}(t),x_{i}(\tau)), \quad t\neq\tau\\ \\
\displaystyle
\frac{1}{2}\ln{\frac{1}{e|x_{i}'(\tau)|^2}}+\tilde{G}(x_{i}(\tau),x_{i}(\tau)), \quad t=\tau.
\end{array}
\right.
\end{equation*}
\hspace*{\parindent}Тепер система (2.8) матиме вигляд:
\begin{equation}
\left\{
\begin{array}{c}
\displaystyle
\frac{1}{2\pi}\int\limits_{0}^{2\pi} \,\mu_1(\tau)\Big[H_{11}(t,\tau)\ln{\Big(\frac{4}{e}\sin^2\frac{t-\tau}{2}\Big)}+H_{12}(t,\tau)\Big]d\tau+\\
\displaystyle
+\frac{1}{2\pi}\int\limits_{0}^{2\pi} \,\mu_2(\tau)L_{1}(t,\tau)d\tau=g_1(t),\quad  t \in [0, 2\pi]\\
\displaystyle
\frac{1}{2\pi}\int\limits_{0}^{2\pi} \,\mu_1(\tau)L_{2}(t,\tau)d\tau+\frac{1}{2\pi}\int\limits_{0}^{2\pi} \,\mu_2 (\tau)\Big[H_{21}(t,\tau)\ln{\Big(\frac{4}{e}\sin^2\frac{t-\tau}{2}\Big)}+\\
\displaystyle
+H_{22}(t,\tau)\Big]d\tau=g_2(t),\quad  t \in [0, 2\pi]
\end{array}
\right.
\end{equation}
\section{Дискретизація}
\hspace*{\parindent}Чисельне розв'язування інтегральних рівнянь в системі (2.9) здійснимо методом квадратур. Для цього на розбитті $t_{j}:=\frac{j\pi}{n}$, $j=0,...,2n-1$, $n\in\mathbb{N}$ розглянемо такі квадратурні формули:
\begin{equation*}
\displaystyle
\frac{1}{2\pi}\int\limits_{0}^{2\pi}f(t)dt\approx
\frac{1}{2n}\sum\limits_{j=0}^{2n-1}f(t_{j}),
\end{equation*}
\begin{equation*}
\displaystyle
\frac{1}{2\pi}\int\limits_{0}^{2\pi}f(\tau)\ln{\frac{4}{e}\sin^2\frac{t-\tau}{2}}d\tau\approx\sum\limits_{j=0}^{2n-1}f(t_{j})R_{j}(t),
\end{equation*}
з ваговими функціями
\begin{equation*}
\displaystyle
R_{j}(t) = -\frac{1}{2n}\Big[1+2\sum\limits_{m=1}^{n-1}{\frac{1}{m}\cos{m(t-t_{j})}} + \frac{1}{n}\cos{n(t-t_{j})}\Big].
\end{equation*}
\hspace*{\parindent}Застосуємо наведені квадратурні правила для апрксимації інтегралів у системі (2.9). В якості точок колокації виберемо квадратурні вузли $t_{j}$, $t_{k}$. Отримаємо наступну систему лінійних рівнянь розмірності $2n\times 2n$:
\begin{equation}
\left\{
\begin{array}{c}
\displaystyle
\sum\limits_{j=0}^{2n-1} \mu_1(t_j)\Big[H_{11}(t_k,t_j)R_j(t_k)+\frac{1}{2n}H_{12}(t_k,t_j)\Big]+\frac{1}{2n}\sum\limits_{j=0}^{2n-1} \mu_2(t_j)L_1(t_k,t_j)=\\
\displaystyle
=g_1(t_k), \quad k=0,...,2n-1\\
\displaystyle
\frac{1}{2n}\sum\limits_{j=0}^{2n-1} \mu_1(t_j)L_2(t_k,t_j)+\sum\limits_{j=0}^{2n-1} \mu_2(t_j)\Big[H_{21}(t_k,t_j)R_j(t_k)+\frac{1}{2n}H_{22}(t_k,t_j)\Big]=\\
\displaystyle
=g_2(t_k), \quad k=0,...,2n-1
\end{array}
\right.
\end{equation}
\section*{Збіжність методу та оцінка похибки}
\begin{theorem}
Похибку складеної формули трапецій:
\begin{equation*}
R_{T}(f)=\frac{1}{2\pi}\int\limits_{0}^{2\pi}f(t)dt-\frac{1}{2n}\sum\limits_{j=0}^{2n-1}f(\frac{j\pi}{n})
\end{equation*}
для аналітичної $2\pi$-періодичної функції $f$ можна оцінити у вигляді:
\begin{equation*}
|R_{T}(f)|\le Ce^{-n\sigma},
\end{equation*}
де $C$, $\sigma>0$ -- константи, залежні від f. 
\end{theorem}
\hspace*{\parindent}Запишемо інтегральні оператори:
\begin{equation*}
(S_{i}\mu)(\tau)=\frac{1}{2\pi}\int\limits_{0}^{2\pi} \,\mu(\tau)\Big[H_{i1}(t,\tau)\ln{\frac{4}{e}\sin^2\frac{t-\tau}{2}}+H_{i2}(t,\tau)\Big]d\tau \quad i=1,2
\end{equation*}
\begin{equation*}
(T_{i}\mu)(\tau)=\frac{1}{2\pi}\int\limits_{0}^{2\pi} \,\mu(\tau)L_{i}(t,\tau)d\tau \quad i=1,2
\end{equation*}
та відповідні їм послідовності квадратурних операторів:
\begin{equation*}
(S_{i,n}\mu)(t)=\sum\limits_{j=0}^{2n-1} \mu(t_{j})\Big[H_{i1}(t,t_j)R_j(t)+\frac{1}{2n}H_{i2}(t,t_j)\Big], \quad i=1,2 \quad t\in[0;2\pi]
\end{equation*}
\begin{equation*}
(T_{i,n}\mu)(t)=\frac{1}{2n}\sum\limits_{j=0}^{2n-1}\mu(t_j)L_{i}(t,t_j), \quad i=1,2, \quad t\in[0;2\pi]
\end{equation*}
\hspace*{\parindent}Нехай оператори $A$ та $A_{n}$ задають системи рівнянь:
\begin{equation*}
A=\left[
\begin{array}{cc}
S_{1}&T_{1}\\
T_{2}&S_{2}
\end{array}
\right]\quad
A_{n}=\left[
\begin{array}{cc}
S_{1,n}&T_{1,n}\\
T_{2,n}&S_{2,n}
\end{array}
\right].
\end{equation*}
\hspace*{\parindent}Розв'язок операторного рівняння
\begin{equation*}
A\mu=F,
\end{equation*}
де 
\begin{equation*}
\mu=\left[
\begin{array}{c}
\mu_{1}\\
\mu_{2}
\end{array}
\right]\quad
F=\left[
\begin{array}{c}
f_{1}\\
f_{2}
\end{array}
\right]
\end{equation*}
апроксимується через розв'язок апроксимаційного рівняння
\begin{equation*}
A_{n}\mu_{n}=F.
\end{equation*}
\begin{theorem}
Для збіжних квадратурних формул послідовність операторів $(A_{n})$ колективно компактна і поточково збіжна:
\begin{equation*}
A_{n}\varphi\rightarrow A\varphi, \ n\to\infty \quad \forall\varphi\in C(\Gamma_{1})\times C(\Gamma_{2}).
\end{equation*}
\end{theorem}
\begin{theorem}
Для достатньо великого n система (2.9) має єдиний розв'язок і має місце оцінка похибки:
\begin{equation*}
\Vert\mu_{n}-\mu\Vert\le M\Vert A_{n}\mu-A\mu\Vert_{\infty}, \quad M>0.
\end{equation*}
\end{theorem}
\hspace*{\parindent}Наближений розв'язок задачі (1.1)-(1.2) шукатимемо у вигляді
\begin{equation*}
u(x)\approx\frac{1}{2n}\sum\limits_{j=0}^{2n-1}{\mu_1(t_j)G(x,x_1(t_j))}+\frac{1}{2n}\sum\limits_{j=0}^{2n-1}{\mu_2(t_j)G(x,x_2(t_j))}-
\end{equation*}
\begin{equation*}
-\sum\limits_{j=0}^{2n-1}{f_0(x_0(t_j))\widehat{G}(x,x_0(t_j))}
\end{equation*}
де 
\begin{equation*}
\widehat{G}(x,x_0(t_j))=\frac{\partial G(x,x_0(t_j))}{\partial\nu(x_0(t_j))}.
\end{equation*}
\chapter{Чисельні експерименти}
\section{Приклад}
\begin{equation*}
\Gamma_0 =\{x_0(t) = (3\cos(t),3\sin(t)), t \in [0,2\pi]\},
\end{equation*}
\begin{equation*}
\Gamma_1 =\{x_1(t) = (\cos(t)+1.5,\sin(t)-1), t \in [0,2\pi]\},
\end{equation*}
\begin{equation*}
\Gamma_2 :=\{x_2(t) = (\cos(t)-1,\sin(t)-1), t \in [0,2\pi]\}.
\end{equation*}
\begin{table}[h]
\caption{\label{tab:pr1}Похибка розв'язку}
\begin{center}
\begin{tabular}{|c|c|c|}
\hline
n & $x(0.8,1.5)$ & $x(-2,0.3)$\\
\hline
4 & $0.037559014$ & $0.067567698$\\
\hline
8 & $0.008565747$ & $0.004356768$\\
\hline
16 & $0.00016735$ & $0.000665354$\\
\hline
32 & $0.00000937$ & $0.000004675$\\
\hline
\end{tabular}
\end{center}
\end{table}
\chapter*{Висновки}
\addcontentsline{toc}{chapter}{Висновки}
\hspace*{\parindent}У роботi дослiджено i знайдено наближений розв'язок крайової задачі Дiрiхле для рiвняння Лапласа в обмеженій області. Це рiвняння має широке застосування в різних розділах фізики, зокрема, у задачах механіки, теплопровідності, електростатики та гідравліці.\\
\hspace*{\parindent}У процесі роботи доведено iснування та єдинiсть класичного розв’язку задачі, зведено задачу до системи iнтегральних рiвнянь першого роду, здійснено параметризацiю, доведено коректнiсть отриманих інтегральних рівнянь у просторах Банаха. Застосовано метод
Нистрьома з використанням потенцiалу простого шару і отримано СЛАР. Iз системи знайдено невiдомi густини, завдяки яким отримано наближений розв’язок поставленої
задачi. Крiм цього доведено збiжнiсть методу i знайдено оцiнку похибки. У роботі наведено приклад, який демонструє збіжність застосованого методу.



\renewcommand{\bibname}{Список літератури}
\addcontentsline{toc}{chapter}{Список літератури}
\begin{thebibliography}{99}

\bibitem{Abramovic}
\emph{Абрамовиц М.} Справочник по специальным функциям / М.Абрамовиц, И.Стиган. - М.:Наука, 1979. - 830 с.

\bibitem{Kolton}
\emph{Колтон Д.} Методы интегральных уравнений в теории рассеяния / Д. Колтон, Р.Крес. - М.:Мир, 1987. - 311 с.

\bibitem{Kress}
\emph{Kress R.} Linear Integral Equations /R.Kress. - Heidelberg:Springer -Velag, 1999. - 368 c.

\end{thebibliography}
\end{document}